
\documentclass[12pt]{article}
\usepackage[T2A]{fontenc}
\usepackage[utf8x]{inputenc}
\usepackage[russian]{babel}
\renewcommand{\baselinestretch}{1.5}
\textwidth=158mm
\textheight=232mm
\voffset=-24mm
\pagestyle{plain} % нумерация вкл.


% Page length commands go here in the preamble
\setlength{\oddsidemargin}{-0.25in} % Left margin of 1 in + 0 in = 1 in
\setlength{\textwidth}{7in}   % Right margin of 8.5 in - 1 in - 6.5 in = 1 in
\setlength{\topmargin}{-.75in}  % Top margin of 2 in -0.75 in = 1 in
\setlength{\textheight}{9.2in}  % Lower margin of 11 in - 9 in - 1 in = 1 in

 
\renewcommand{\baselinestretch}{1.5} % 1.5 denotes double spacing. Changing it will change the spacing

\setlength{\parindent}{0in} 


\begin{document}
\title{WKTruckputer: концепция интерфейса}
\author{Reiner}
\date{\today}
\maketitle
\abstract{Описание концепции интерфейса. Далее документ будет делаться всё подробнее}


	\section {Графическая ололочка}
		Организуется следующий режим: по умолчанию рабочий стол штатный. Можно что-то открыть, пользоваться софтом и так далее. Но по нажатию хардверных кнопок происходит вызов обозначенных программ либо переход к их окнам, если они уже вызваны. Окна обязательно раскрываются на весь экран. Системными командами, впрочем, окно можно свернуть, изменить его размер, перенести в левую, правую, верхнюю, нижнюю половину экрана, поставить tile (что на мелком экране почти бесполезно, да).  При двух одновременно нажатых хардверных кнопках мы запускаем обе программы, и разносим окна на две половины экрана - правую левую. Три одновременно нажатые кнопки не должны делать ничего. Ещё одна кнопка должна полностью зачищать экран. Также кнопками или крутилкой должна регулироваться громкость, пауза, вперёд-назад у плейера. Суммарно выходит около 20 кнопочек. 
	\section {Какие программы хорошо видеть на этом быстром запуске }
		\begin {enumerate}
		
		\item \emph{GPS} Navit, GPSDrive
		\item \emph{Music} mocp 
		\item \emph{Браузер}
		\item \emph{ODB}
		\item \emph{Консоль}
		\item \emph{Файловый браузер}
		\item \emph{Латеховый редактор :-) }
		\item \emph{Система видеорегистрации }
		\item \emph{Электронная почта}
		\item \emph{Датчики различных параметров (вроде потребления тока и так далее)}
		\end {enumerate}
		
	\section{Хардверные решения }
		\subsection{Клавиатура}
		Берём контроллер, подключаем к нему резистивную клавиатуру. Сам контроллер подключаем к нашему компьютеру через UART, SPI или ещё какой-нибудь интерфейс. 
	Получаются удобные хардверные кнопки. В дальнейшем можно также поставить джойстик (jog dial), различные рычажки и регуляторы. Резисторы надлежит подбирать так, чтоб две нажатых кнопки никогда не были эквивалентны одной. 
		
	\section{Подсистемы}
	\subsection {Система видеорегистрации}
	Её основной функционал - она должна записывать видеопоток с каждой камеры в кольцевой буфер. Сжимать видео "на лету". Обеспечивать возможность по нажатию пары кнопок скидывать видео на флешку или карточку. Добавлять на видео скорость и координаты. Записывать звук. Делаться это в примитивном варианте может через вебкамеру, в продвинутом - через линейный видеовход на кубе. Либо часть через вебкамеру, часть - через линейный видеовход. \\
	
	
	




\end{document}